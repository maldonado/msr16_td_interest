
\section{Initial Case Study} \label{sec:results}
\smallsection{Motivation}
There exist several previous studies that focused on understanding SATD (e.g., the detection of technical debt~\cite{Potdar2014ICSME,Zazworka2013EASE} and the impact of SATD on software quality~\cite{Wehaibi2016SANER}). However, to the best of our knowledge, there are no studies that help in the quantification of SATD interest. Therefore, we would like to know how we can measure interest and if SATD actually incurs positive interest.

\smallsection{Datasets}
To conduct our initial case study, we use data from the Apache JMeter open source. We use JMeter since we have used this dataset in the past~\cite{Maldonado2015MTD,Potdar2014ICSME}, and know that it cotains instances of SATD and uses Git as the version control system, which many of our tools are designed to work on. In particular, we use release v2.10 of JMeter, which contains 81,307 SLOC in 1,181 classes, contains 20,084 comments, and has 33 unique contributors.
% Table \ref{tab:project} shows the statistics of the project we use in our experiments. 

%\begin{table}[tb]
%  \caption{Project details}
%  \label{tab:project}
%  \centering

%  \begin{tabular}{l|rrrp{1.3cm}p{1.3cm}}
%  \hline
%    Project & Release & \# of classes & SLOC & \# of comments & \# of contributors \\
%  \hline
%    JMeter & {\sc v2\_10} &   1,181  &  81,307  & 20,084  &  33 \\
%  \hline
%  \end{tabular}
%\end{table}

%\todo{How do we choose projects we analyze? i.e., why do we use Ant and Jmeter and do not use ArgoUML, Columba and JFreeChart? and why do we add jRuby?}


\smallsection{Approach}
%\para{We calculate the interest.}
To calculate interest of SATD, we follow the approach we explained in Section \ref{sec:approach}.
We show the number of SATD, the percentage of the technical debt that has positive interest, and the distribution of interest for technical debt that incurs an positive interest rate.

\smallsection{Results}
We find that there is a high correlation between LOC and the other product metrics, except Fan-In. From the highly correlated metrics, we selected LOC as the metric to calculate interest, since intuitively it is easier to measure and comprehend. Therefore, we settled on using two product metrics (i.e., LOC and Fan-In) to measure interest.

%similar to previous work that considers effort in the domain of defect prediction~\cite{Kamei2010ICSM,Kamei2013TSE}. We assume that developers spend more effort to check larger methods before modifying the methods. Eventually, we show our results using 

Table \ref{tab:percentage} shows the number of SATD and the percentage of the technical debt that has positive interest in all technical debt. The table shows that 44.2\% of technical debt incurs a positive interest rate in terms of LOC and 42.2\% of the SATD has it in terms of Fan-In. We can see that in some cases, there can be negative interest (13.8\% using LOC and 8.1\% using Fan-In), where the SATD method gets smaller or have less Fan-In after the introduction of the SATD. There are also cases where nothing changes in terms of LOC and Fan-In between the SATD introduction and removal. Lastly, it is important to note that there is not large difference between in the amount of positive and no change interest rates using LOC and Fan-In. 

Next, we would like to know how high is the positive interest rate. This analysis provides us with more insight about the SATD that incurs a positive interest rate.
Table \ref{tab:statistic} and Figure \ref{fig:dist} show that the distribution of interest for the SATD that incurs a positive rate. We see from Figure \ref{fig:dist}, that the distributions are left-skewed, indicating that the majority of the SATD ranges between 6.5-11.0 and 33.3 in terms of LOC and Fan-In. Our findings clearly indicate that there is SATD that incurs a positive interest rate and different types of SATD have different values of interest, which shows that we should be prioritizing SATD based on its interest, i.e., all SATD is not equal.


%We can also find that some of positive interest are over 100, which means the value of metric relatively increases by 100\%.

\begin{table}[tb]
  \caption{The Percentage of SATD that has Positive, Negative and No Change in Interest}
  \label{tab:percentage}
  \centering

  \begin{tabular}{c|r|rrrr}
  \hline
        & \textbf{Positive Rate} & \textbf{All} & \textbf{Positive} & \textbf{Negative} & \textbf{No Change} \\
  \hline
   LOC  & 44.2\% & 181 &  80  &  25 & 76\\
Fan-In  & 42.2\% & 161 &  68  &  13 & 80\\
  \hline
  \end{tabular}
\end{table}

\begin{table}[tb]
  \caption{Statistics}
  \label{tab:statistic}
  \centering

  \begin{tabular}{c|rrrrr}
  \hline
        & \textbf{Min.} & \textbf{1st Qu.} & \textbf{Median} & \textbf{3rd Qu.} & \textbf{Max.} \\
  \hline
   LOC  & 1.5 &   6.5 &  15.3  &   33.3 &   98.5 \\
Fan-In  & 5.3 &  11.0 &  20.0  &   33.3 &   90.0 \\
  \hline
  \end{tabular}
\end{table}
% this table is from 1825c645ba885f9aeb48779130095763dddeaccb

\conclusionbox{
44.2\% of technical debt incurs a positive rate in terms of LOC and 42.2\% of technical debt incurs it in terms of Fan-In.}


%\para{Put additional analysis when considering time period.}

%\para{Put additional analysis when considering other metrics (fan-in).}

%\para{Put the analysis for showing the method that includes more than one technical debt in one version.}