%What is technical debt and self-admitted technical debt
\section{Introduction}
Technical debt was first coined by Cunningham in 1993 to refer to the phenomena of taking a shortcut to achieve short term development gain at the cost of increased maintenance effort in the future \cite{Cunningham1992WPM}. The technical debt community, organized through the managing technical debt workshop \cite{MTD2016}, has studied many aspects of technical debt, including its detection \cite{Zazworka2013EASE}, impact \cite{Zazworka2011MTD} and the appearance of technical debt in the form of code smells \cite{Fontana2012MTD}. Most recently, we developed an approach to identify technical debt from code comments, referred to as self-admitted technical debt (SATD). SATD refers to the situation where developers know that the current implementation is not optimal and write comments alerting the inadequacy of the solution. 

% What people did and what is the impact of TD. What they found.
In the last few years, an increasing amount of work has focused on SATD. In particular, our prior work focused on the detection of SATD~\cite{Potdar2014ICSME} and the classification of different types of SATD and the development of datasets to enable future studies on SATD~\cite{Maldonado2015MTD}. Other work by Bavota and Russo~\cite{Bavota2016MSR} performed an empirical study of SATD on a large number of Apache projects showed that SATD is prevalent in open source projects, is long lived and is increasing over time. A study by Wehaibi et al.~\cite{Wehaibi2016SANER} examined the impact of SATD on quality and found that SATD does not necessarily relate to more defects, however, it does make the software system more complex. 

%However, very little work focused on interest. Also, why is calculating interest difficult
Although the metaphor of technical debt has been well studied, to the best of our knowledge, the cost of debt/interest has not been previously studied. Measuring the interest of the technical debt is one of the challenges in the field, since it requires for the detection of the technical debt, the tracking of the debt over time and the development of measures to accurately quantify this debt. Given that SATD allows us to know the exact method the technical debt exists in, we are able to perform fine-grained analysis of the code, which enables us to quantify interest of the debt.

In this paper, we first propose the use of code metrics, in particular the well-known Lines of Code (LOC) and Fan-In, to measure interest. We use LOC since it highly correlates with most code complexity metrics and Fan-In since it allows us to measure how much a piece of code is depended on by other code. Then, we use the developed measure to determine how much of the SATD incurs positive interest. In a case study on the Apache JMeter project, we find that using LOC, 44.2\% and using Fan-In 42.2\% of the SATD in JMeter incurs positive interest.

% Organization of the paper
The rest of the paper is organized as follows; Section \ref{sec:approach} introduces our approach to quantify interest of SATD. Section \ref{sec:results} describes a preliminary study using the developed measure. Finally, Section \ref{conclusion} draws conclusions and our future work.
