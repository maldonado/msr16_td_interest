%What is technical debt and self-admitted technical debt
Technical debt is term first coined by Cunningham in 1993 to refer to the phenomena of taking a shortcut to achieve short term development gain at the the cost of increased maintenance effort in the future \cite{Cunningham1992WPM}. The technical debt community, organized through the managing technical debt workshop \cite{Falessi2014MTD}, has studied many aspects of technical debt, including its detection \cite{Zazworka2013CSE}, impact \cite{Zazworka2011MTD}
and the appearance of technical debt in the form of code smells \cite{Fontana2012MTD}. Most recently, we developed an approach to identify technical debt from code comments, referred to as self-admitted technical debt (SATD). SATD refers to the situation where developers know that the current implementation is not optimal and write comments alerting the inadequacy of the solution.

% What people did and what is the impact of TD. What they found.
In the last few years, an increasing amount of work has focused on SATD. In particular, our prior work focused on the detection of SATD~\cite{Potdar2014ICSME} and the classification of different types of SATD and the development of datasets to enable future studies on SATD~\cite{Maldonado2015MTD}. Other work by Bavota and Russo performed an empirical study of SATD on a large number of Apache projects showed that SATD is prevalent in open source projects, is long lived and is increasing over time. A study by Wehaibi et al. \todo{cite Wehaibi} examined the impact of SATD on quality and found that SATD does not necessarily relate to more defects, however, it does make the software system more complex. 

%However, very little work focused on interest. Also, why is calculating interest difficult
Based on these prior findings, we measure and quantify the effect of SATD. In particular, we measure the amount of \emph{interest} caused by SATD. Although the metaphor of technical debt has been well studied, to the best of our knowledge, the quanitification of interest of technical debt has not been examined before. Measuring the interest if technical debt is non-trivial since it requires, in addition to the detection of the technical debt, the tracking of the debt over time and the development of a measure to accurately quantify this debt.

% What we do  and how we calculate interest
In this paper, we first propose the use of code metrics, in particular \todo{add}, as a measure of interest. We 
 
% Main contributions
The main contributions of the paper are three-fold.

\begin{itemize}
\end{itemize}

% Organization of the paper
\smallsection{Paper Organization} To purse the goal of this paper, the paper is organized as follows. 
Section \ref{background} explains the overview of defect prediction models.
Section \ref{past} revisits what challenges were in Year 2000.
%Section \ref{trends} assesses what state the research trend is in.
%Section \ref{game_changers} presents game changers, which dramatically changed perspective and direction of the studies in the field of defect prediction.
Section \ref{trends} assesses what state the research trend is in and presents some of game changers, which dramatically changed perspective and direction of the studies in the field of defect prediction.
Section \ref{challenges} highlights some key challenges for future.
Section \ref{conclusion} draws conclusions.
