\documentclass[conference]{IEEEtran}
%\usepackage{graphicx}
\usepackage[dvipdfmx]{graphicx}
%\usepackage{latexsym}
\usepackage{slashbox}
%\usepackage{multirow}
\usepackage{url}
\usepackage{color}
\usepackage{colortbl}
\usepackage{hhline}
\usepackage{flushend}
\usepackage{verbatim}
%\usepackage{hyperref}
\usepackage{enumerate}
\usepackage[sort&compress]{natbib}
\usepackage{subfigure}
\usepackage{framed}
%\usepackage{natbib}

\newcommand{\todo}[1]{{\color{green}{\textbf{TODO: [#1]}}}}
\newcommand{\yasu}[1]{{\color{red}{\textbf{Yasu says: [#1]}}}}
\newcommand{\emad}[1]{{\color{blue}{\textbf{[#1]}}}}
\newcommand{\Emad}[1]{{\color{blue}{\textbf{[#1]}}}}
\newcommand{\revised}[1]{{\color{red}{#1}}}
\newcommand{\para}[1]{{\color{magenta}{\textbf{This paragraph:}}} [#1]}

%\newcommand{\revised}[2]{\marginpar{\fbox{#2}}{\color{red}{#1}}}

\newcommand{\nbf}[1]{
%  \noindent{\textit{\textbf{#1}}}
  \noindent{\textbf{#1}}
}

\newcommand{\conclusionbox}[1]{%
	\vspace{2mm}
  \noindent
	\framebox[0.48\textwidth][c]{%
		\parbox[b]{0.45\textwidth}{%
			{\em #1}
		}
	}
}

% Set bibliography title
%\renewcommand\refname{REFERENCES}
%\renewcommand\bibsection{\section{\refname}}

\newcommand{\ea}{{\em et al.}}
\newcommand{\smallsection}[1]{\vspace{1mm}\noindent {\bf #1}.\hspace{2mm}}
\newcommand{\emphsection}[1]{\vspace{1mm}\noindent \underline{{\em #1}.}\hspace{2mm}}


% Reduce bibliography font size
%\def\bibfont{\normalsize} %normalsize should be default
% small or footnotesize

% Reduce space between references
%\setlength{\bibsep}{3.2pt} %3.5pt should be default


\begin{document}

\title{How Much do We Pay for Technical Debt as Interest?}

\author{
\IEEEauthorblockN{Yasutaka Kamei$^{\dag}$, Everton Maldonado$^{\dag\dag}$, Emad Shihab$^{\dag\dag}$, and Naoyasu Ubayashi$^{\dag}$}
\IEEEauthorblockA{
$^{\dag}$Principles of Software Languages Group (POSL), Kyushu University, Fukuoka, Japan\\
$^{\dag\dag}$Department of Computer Science and Software Engineering, Concordia University, Montr\'eal, Canada\\
Email: $^{\dag}$\{kamei, ubayashi\}@ait.kyushu-u.ac.jp, $^{\dag\dag}$\{e\_silvam, eshihab\}@encs.concordia.ca
}
}

% make the title area
\maketitle

% As a general rule, do not put math, special symbols or citations
% in the abstract
\begin{abstract}
abstract.
\end{abstract}

\IEEEpeerreviewmaketitle

% Here is Yasu't note
\begin{comment}
2 tomato: let's finish results (how to show)
2 tomato: read related work to get knowledge and good motivation for RQ1 and RQ2.

- slide:
2 tomato (one topic of current)
2 tomato (one topic of current)
\end{comment}

%%%%%%%%%%%%%%%%%%%%%%%%%%%%%%%%%%%%%%%%%%%%%%%%%%%%%%%%%%%
%\section{Introduction}

\emph{If you know your enemies and know yourself, you will not be imperiled in a hundred battles~\cite{WikiquoteSunTzu}.} This is the quote by Sun Tzu (c. 6th century BCE), who was a Chinese general, military strategist, and author of the book The Art of War, an immensely influential ancient Chinese book on military strategy. This quote is the one of the principle of empirical software engineering. To know your \emph{enemies} (i.e., bugs) and \emph{yourself} (i.e., software systems) and win \emph{battles} (i.e., leading a project to success conclusion), one needs to investigate a large amount of research on Software Quality Assurance (SQA). SQA can be broadly defined as the set of activities that ensure software meets a specific quality level~\cite{Fenton1999TSE}.

%\para{we explain why software is important. Everybody use software as not only infrastructure but also daily wants (e.g., mobile apps). We also show how much money are lost by bugs. } \todo{Need to find some papers that say how much money are lost by bugs in the domain of infra and mobile apps.}

As software systems continue to play an increasingly important role in our lives, their complexity continues to increase; making SQA efforts very difficult. At the same time, the importance of SQA efforts is of paramount importance, as shown by the US National Institute of Standards and Technology (NIST) study, which estimated that software faults and failures cost the US economy \$59.5 billion a year~\cite{NIST}. Therefore, to ensure high software quality, software defect prediction models, which describe the relationship between various software metrics (e.g., SLOC and McCabe's Cyclomatic complexity) and software defects, have been proposed~\cite{Zimmermann2007,Moser2008ICSE}. Traditionally, the software defect prediction models are used in two ways: (1) to predict where defects might appear in the future and allocate SQA resources to defect-prone artifacts (e.g., subsystems and files)~\cite{Munson1992} and (2) to understand the effect of factors on the likelihood of finding a defect and derive practical guidelines for future software development projects~\cite{Cataldo2009TSE,McIntosh2014MSR}.

%We still eagerly focus on SQA studies, software systems are becoming complexity and being used in everything from mobile devices to artificial satellites. The increasing importance and complexity make the quality of software systems critical and rise the cost of SQA activities. For example, the  Since a company has only limited resources (e.g., developers and cost) for SQA activities, these activities have to be performed as efficiently as possible.

% http://blog.typemock.com/2012/05/what-is-the-cost-of-avoiding-unit-testing-and-the-cost-of-software-bugs.html

%\para{To know bugs, many studies about software quality assurance (SQA) are conducted, especially using empirical approaches.} We introduce one of oldest empirical papers and a couple of papers? Bug prediction (defect prediction. We define what bug prediction is.) \yasu{I would like to explain not only ``prediction'', but also ``understanding'' in our context}.

%One line of work that have received a lot of attention in recent years is defect prediction models, which describe the relationship between factors (e.g., SLOC and McCabe's Cyclomatic complexity) as predictor variables and a status (e.g., defect-prone or not after releases) as a response variable~\cite{Ohlsson1996TSE,Zimmermann2007,Moser2008ICSE}. The models can be used as (1) to predict where defects might appear in the future and allocate SQA resources to defect-prone artifacts (e.g., subsystems and files)~\cite{Munson1992, } and (2) to understand the effect of factors on the likelihood of finding a defect and derive practical guidelines for future software development projects~\cite{Cataldo2009TSE,McIntosh2014MSR}.

Due to its importance, defect prediction work has been at the focus of researchers for over 40 years. As Fenton and Neil~\cite{Fenton2000ICSE} explained, Akiyama~\cite{Akiyama1971IFIP} first attempted to build defect prediction models using size-based metrics and regression modelling techniques in 1971. Since then, there have been many studies and many accomplishments in software defect prediction. At the same time, there remain many challenges that face software defect prediction. Hence, we find that it is a perfect time to write a Future of Software Engineering (FoSM) paper on the topic of software defect prediction.

The paper is written from a budding university researchers' point of view and aims to accomplish \todo{three} things. First, we provide a brief overview of software defect prediction and its various components. Second, we revisit the challenges of software prediction models as they were seen in the year 2000, in order to reflect on our accomplishments since then. We also highlight our accomplishments and recent trends, as well as, discuss the game changers that had a significant impact on software defect prediction. Third, we highlight some key challenges that lie ahead in the near (and not so near) future in order to us as a research community to tackle these future challenges.

%his areaAfter the first publication, there are many studies on defect prediction models~\cite{Ohlsson1996TSE,Zimmermann2007,Moser2008ICSE}. In addition, thanks to the rich and readily available software repositories from modern software development environments, the defect prediction model filed is accelerating research speed and uncovering useful information about software projects.

%However, we still have many challenges in the field of software defect prediction. In this paper, we would like to highlight some key challenges for future (from budding university researchers' point of view). 

\begin{figure*}
  \centering
  \includegraphics[trim=0 60 0 80, scale=0.5,clip] {figures/overview}
  \caption{Overview of Software Defect Prediction (SDP)~\cite{Shihab2012PhD} \label{fig:overview}}
\end{figure*}

\smallsection{Target Readers} We hope that all researchers and practitioners, especially masters students, PhD students and young researchers, read this paper in order to gain an understanding of whole pictures of defect prediction studies and some key challenges for future, then share their opinion and discuss the future of software quality assurance with us.

\smallsection{Paper Organization} To purse the goal of this paper, the paper is organized as follows. 
Section \ref{background} explains the overview of defect prediction models.
Section \ref{past} revisits what challenges were in Year 2000.
%Section \ref{trends} assesses what state the research trend is in.
%Section \ref{game_changers} presents game changers, which dramatically changed perspective and direction of the studies in the field of defect prediction.
Section \ref{trends} assesses what state the research trend is in and presents some of game changers, which dramatically changed perspective and direction of the studies in the field of defect prediction.
Section \ref{challenges} highlights some key challenges for future.
Section \ref{conclusion} draws conclusions.

\section{Introduction}

\section{Introduction}

\emph{If you know your enemies and know yourself, you will not be imperiled in a hundred battles~\cite{WikiquoteSunTzu}.} This is the quote by Sun Tzu (c. 6th century BCE), who was a Chinese general, military strategist, and author of the book The Art of War, an immensely influential ancient Chinese book on military strategy. This quote is the one of the principle of empirical software engineering. To know your \emph{enemies} (i.e., bugs) and \emph{yourself} (i.e., software systems) and win \emph{battles} (i.e., leading a project to success conclusion), one needs to investigate a large amount of research on Software Quality Assurance (SQA). SQA can be broadly defined as the set of activities that ensure software meets a specific quality level~\cite{Fenton1999TSE}.

%\para{we explain why software is important. Everybody use software as not only infrastructure but also daily wants (e.g., mobile apps). We also show how much money are lost by bugs. } \todo{Need to find some papers that say how much money are lost by bugs in the domain of infra and mobile apps.}

As software systems continue to play an increasingly important role in our lives, their complexity continues to increase; making SQA efforts very difficult. At the same time, the importance of SQA efforts is of paramount importance, as shown by the US National Institute of Standards and Technology (NIST) study, which estimated that software faults and failures cost the US economy \$59.5 billion a year~\cite{NIST}. Therefore, to ensure high software quality, software defect prediction models, which describe the relationship between various software metrics (e.g., SLOC and McCabe's Cyclomatic complexity) and software defects, have been proposed~\cite{Zimmermann2007,Moser2008ICSE}. Traditionally, the software defect prediction models are used in two ways: (1) to predict where defects might appear in the future and allocate SQA resources to defect-prone artifacts (e.g., subsystems and files)~\cite{Munson1992} and (2) to understand the effect of factors on the likelihood of finding a defect and derive practical guidelines for future software development projects~\cite{Cataldo2009TSE,McIntosh2014MSR}.

%We still eagerly focus on SQA studies, software systems are becoming complexity and being used in everything from mobile devices to artificial satellites. The increasing importance and complexity make the quality of software systems critical and rise the cost of SQA activities. For example, the  Since a company has only limited resources (e.g., developers and cost) for SQA activities, these activities have to be performed as efficiently as possible.

% http://blog.typemock.com/2012/05/what-is-the-cost-of-avoiding-unit-testing-and-the-cost-of-software-bugs.html

%\para{To know bugs, many studies about software quality assurance (SQA) are conducted, especially using empirical approaches.} We introduce one of oldest empirical papers and a couple of papers? Bug prediction (defect prediction. We define what bug prediction is.) \yasu{I would like to explain not only ``prediction'', but also ``understanding'' in our context}.

%One line of work that have received a lot of attention in recent years is defect prediction models, which describe the relationship between factors (e.g., SLOC and McCabe's Cyclomatic complexity) as predictor variables and a status (e.g., defect-prone or not after releases) as a response variable~\cite{Ohlsson1996TSE,Zimmermann2007,Moser2008ICSE}. The models can be used as (1) to predict where defects might appear in the future and allocate SQA resources to defect-prone artifacts (e.g., subsystems and files)~\cite{Munson1992, } and (2) to understand the effect of factors on the likelihood of finding a defect and derive practical guidelines for future software development projects~\cite{Cataldo2009TSE,McIntosh2014MSR}.

Due to its importance, defect prediction work has been at the focus of researchers for over 40 years. As Fenton and Neil~\cite{Fenton2000ICSE} explained, Akiyama~\cite{Akiyama1971IFIP} first attempted to build defect prediction models using size-based metrics and regression modelling techniques in 1971. Since then, there have been many studies and many accomplishments in software defect prediction. At the same time, there remain many challenges that face software defect prediction. Hence, we find that it is a perfect time to write a Future of Software Engineering (FoSM) paper on the topic of software defect prediction.

The paper is written from a budding university researchers' point of view and aims to accomplish \todo{three} things. First, we provide a brief overview of software defect prediction and its various components. Second, we revisit the challenges of software prediction models as they were seen in the year 2000, in order to reflect on our accomplishments since then. We also highlight our accomplishments and recent trends, as well as, discuss the game changers that had a significant impact on software defect prediction. Third, we highlight some key challenges that lie ahead in the near (and not so near) future in order to us as a research community to tackle these future challenges.

%his areaAfter the first publication, there are many studies on defect prediction models~\cite{Ohlsson1996TSE,Zimmermann2007,Moser2008ICSE}. In addition, thanks to the rich and readily available software repositories from modern software development environments, the defect prediction model filed is accelerating research speed and uncovering useful information about software projects.

%However, we still have many challenges in the field of software defect prediction. In this paper, we would like to highlight some key challenges for future (from budding university researchers' point of view). 

\begin{figure*}
  \centering
  \includegraphics[trim=0 60 0 80, scale=0.5,clip] {figures/overview}
  \caption{Overview of Software Defect Prediction (SDP)~\cite{Shihab2012PhD} \label{fig:overview}}
\end{figure*}

\smallsection{Target Readers} We hope that all researchers and practitioners, especially masters students, PhD students and young researchers, read this paper in order to gain an understanding of whole pictures of defect prediction studies and some key challenges for future, then share their opinion and discuss the future of software quality assurance with us.

\smallsection{Paper Organization} To purse the goal of this paper, the paper is organized as follows. 
Section \ref{background} explains the overview of defect prediction models.
Section \ref{past} revisits what challenges were in Year 2000.
%Section \ref{trends} assesses what state the research trend is in.
%Section \ref{game_changers} presents game changers, which dramatically changed perspective and direction of the studies in the field of defect prediction.
Section \ref{trends} assesses what state the research trend is in and presents some of game changers, which dramatically changed perspective and direction of the studies in the field of defect prediction.
Section \ref{challenges} highlights some key challenges for future.
Section \ref{conclusion} draws conclusions.

%\para{What is technical debt?}
%
%\para{Describe more detail of technical debt and current problem in the domain}
%
%\para{The goal and approach of this study}
%
%\para{Contributions}
%
%\cite{Potdar2014ICSME}
%\cite{Maldonado2015MTD}

%%%%%%%%%%%%%%%%%%%%%%%%%%%%%%%%%%%%%%%%%%%%%%%%%%%%%%%%%%%
%\section{Background} \label{background}
%\emph{Prevention is better than cure.}
\para{Technical debt}

\cite{Guo2011ICSM}: they track technical debt items and access its impact (cost) among incurred, deferred and paid. That said, they focused on only one event (WebDav protocol is not supported) and 

\para{Software Evolution: We calculate interest by looking at the difference size of two versions. In other words, this work is one of lines of software evolution.}
\section{Related Work}
\para{Technical debt}

\cite{Guo2011ICSM}: they track technical debt items and access its impact (cost) among incurred, deferred and paid. That said, they focused on only one event (WebDav protocol is not supported) and 

\para{Software Evolution: We calculate interest by looking at the difference size of two versions. In other words, this work is one of lines of software evolution.}

%%%%%%%%%%%%%%%%%%%%%%%%%%%%%%%%%%%%%%%%%%%%%%%%%%%%%%%%%%%
\section{Case Study Setup}
The goal of this study is \todo{purpose.}.
...

We formalize our study in the following research questions:

\todo{List up research question}

To conduct our case study, we use data from ...

\todo{How do we choose projects we analyze? i.e., why do we use Ant and Jmeter and do not use ArgoUML, Columba and JFreeChart? and why do we add jruby?}

Similar to previous work~\cite{Kamei2010ICSM,Kamei2013TSE}, we used churn as effort. We assume that developers spend their effort to check the method before modifying methods at least, we say that is the effort.

\subsection{Project Data Extraction}
\para{Get Git repos}

\para{}

\subsection{Interest}
To measure interest, we use source code metrics. We use Understand, which is a well-known tool, to measure source code metrics. 

\para{how to calculate metrics?}

In this study, we consider the relative size of ... as interest. For example, 

\para{how to identify when a technical debt are introduced and removed?}
From that version, we obtain patches between two versions over all versions for each file that includes technical debt. Then, we check each patch about whether or not technical debt is introduced/removed. Some of them are not removed. If so, we use latest commits. We calculate the difference of a metric between two versions when technical debt is introduced and removed.

%%%%%%%%%%%%%%%%%%%%%%%%%%%%%%%%%%%%%%%%%%%%%%%%%%%%%%%%%%%
\section{Results}
\subsection{RQ1: Can we quantify interest of technical debt at the function-level?}
\smallsection{Motivation}
\todo{need to read some papers that say technical debt is bad. So we want to quantify it.}

\smallsection{Approach}
\para{We calculate the interest.}

\smallsection{Results}

Put Table.

Figure xx shows that the distribution of interest. X
\para{Put the distribution of interest?}

割合と分布について述べる.

%-----------------------------------------------------------------------
\begin{figure}
  \centering
  \includegraphics[width=.45\textwidth]{figures/rq1-ant}
  \caption{Adding a Repository in Commit Guru \label{fig:guru1}}
\end{figure}
\begin{figure}
  \centering
  \includegraphics[width=.45\textwidth]{figures/rq1-jmeter}
  \caption{Adding a Repository in Commit Guru \label{fig:guru1}}
\end{figure}
\begin{figure}
  \centering
  \includegraphics[width=.45\textwidth]{figures/rq1-jruby}
  \caption{Adding a Repository in Commit Guru \label{fig:guru1}}
\end{figure}
%-----------------------------------------------------------------------


\conclusionbox{Result of RQ1: 22\%-36\% of technical debt has positive interest.}

\subsection{RQ2: Does the interest differ based on the type of technical debt?}
\smallsection{Motivation}
There are several type of technical debt such as defect technical debt and design technical debt. 
To better understand the interest, we would like to analayze ...

\smallsection{Approach}
We classify technical debt into categories and calculate interest in each category.

\smallsection{Results}
Table X shows the number and the percentage of technical debt. Among three projects, jruby only includes more than one category that has more than 10\% of technical debt methods. Therefore, we decided to use jruby in RQ2.

\para{we report same things in RQ1.}  

%-----------------------------------------------------------------------
\begin{figure}
  \centering
  \includegraphics[width=.45\textwidth]{figures/rq2-defect}
  \caption{Adding a Repository in Commit Guru \label{fig:guru1}}
\end{figure}
\begin{figure}
  \centering
  \includegraphics[width=.45\textwidth]{figures/rq2-design}
  \caption{Adding a Repository in Commit Guru \label{fig:guru1}}
\end{figure}
\begin{figure}
  \centering
  \includegraphics[width=.45\textwidth]{figures/rq2-requirement}
  \caption{Adding a Repository in Commit Guru \label{fig:guru1}}
\end{figure}
%-----------------------------------------------------------------------


\conclusionbox{Result of RQ2}

\subsection{RQ3: manual analysis}
\smallsection{Motivation}

\smallsection{Approach}

\smallsection{Results}

\conclusionbox{Result of RQ3}

\section{Discussion}
\para{Put additional analysis when considering time period.}

\para{Put additional analysis when considering other metrics (fan-in).}

\para{Put additional analysis when comparing the non-technical deb methods.}

\para{Put the analysis for showing the method that includes more than one technical debt in one version.}

%%%%%%%%%%%%%%%%%%%%%%%%%%%%%%%%%%%%%%%%%%%%%%%%%%%%%%%%%%%
\section{Threat to Validity}

\smallsection{Construct Validity}

\smallsection{Internal Validity}
\para{Technical debt is classified by one person.}

\smallsection{External Validity}

%%%%%%%%%%%%%%%%%%%%%%%%%%%%%%%%%%%%%%%%%%%%%%%%%%%%%%%%%%%
\section{Conclusion}
%\section{Conclusion} \label{conclusion}
%\section{Conclusion} \label{conclusion}
%\section{Conclusion} \label{conclusion}
%\input{conclusion.tex}
In this paper, we introduce the approach to quantify interest of SATD. Our proposed approach uses
software metrics to lead to automated ways to measure the interest from large source code base. 
The results of our initial case study using the Apache JMeter projet
show that 44.2\% of technical debt has a positive rate in terms of LOC and 42.2\% of technical debt has it in terms of Fan-In.

\smallsection{Future work} This paper shows only early idea to quantity interest of SATD. Therefore, there remain
many challenges in this topic. 

\begin{itemize}
\item To calculate interest, we use the relative size of metric values between two versions of SATD-introduction and reduction. However, the period is not considered to calculate the interest. Therefore, we would like to take the period into account when calculating the interest.
\item  There are several type of technical debt such as defect technical debt and design technical debt.
The previous study~\cite{Maldonado2015MTD} shows that the percentage of technical debt varies depending on the type of technical debt and the studied systems. For example, the projects that have limited time to develop features are likely to leave comments of features that need to be implemented in the future. 
To better understand the interest, we would like to analyze the interest per type of technical debt.
\item  The interest varies among technical debt. If we can understand the reason why some of technical debt has large interest, we can make use of such insights for future development. Therefore, we would like to manually investigate why some of technical debt has large interest.
\item Generally speaking, software systems are always evolving over time for implementing new functionality and fixing defects.
Therefore, even if the size of technical debt increases, it is not clear about how the nature of software evaluation affects the interest of technical debt.
We would like to compare the impact of software evolution on methods in two groups of SATD v.s. non-SATD.
%\item To operationalize our findings, we also built a tool that is able to identify and assign an interest rate to all SATD instances in a project. Our tool is publicly available and can be used by practitioners to prioritize the most impacting (i.e., highest interest) SATD.
\end{itemize}

In this paper, we introduce the approach to quantify interest of SATD. Our proposed approach uses
software metrics to lead to automated ways to measure the interest from large source code base. 
The results of our initial case study using the Apache JMeter projet
show that 44.2\% of technical debt has a positive rate in terms of LOC and 42.2\% of technical debt has it in terms of Fan-In.

\smallsection{Future work} This paper shows only early idea to quantity interest of SATD. Therefore, there remain
many challenges in this topic. 

\begin{itemize}
\item To calculate interest, we use the relative size of metric values between two versions of SATD-introduction and reduction. However, the period is not considered to calculate the interest. Therefore, we would like to take the period into account when calculating the interest.
\item  There are several type of technical debt such as defect technical debt and design technical debt.
The previous study~\cite{Maldonado2015MTD} shows that the percentage of technical debt varies depending on the type of technical debt and the studied systems. For example, the projects that have limited time to develop features are likely to leave comments of features that need to be implemented in the future. 
To better understand the interest, we would like to analyze the interest per type of technical debt.
\item  The interest varies among technical debt. If we can understand the reason why some of technical debt has large interest, we can make use of such insights for future development. Therefore, we would like to manually investigate why some of technical debt has large interest.
\item Generally speaking, software systems are always evolving over time for implementing new functionality and fixing defects.
Therefore, even if the size of technical debt increases, it is not clear about how the nature of software evaluation affects the interest of technical debt.
We would like to compare the impact of software evolution on methods in two groups of SATD v.s. non-SATD.
%\item To operationalize our findings, we also built a tool that is able to identify and assign an interest rate to all SATD instances in a project. Our tool is publicly available and can be used by practitioners to prioritize the most impacting (i.e., highest interest) SATD.
\end{itemize}

In this paper, we introduce the approach to quantify interest of SATD. Our proposed approach uses
software metrics to lead to automated ways to measure the interest from large source code base. 
The results of our initial case study using the Apache JMeter projet
show that 44.2\% of technical debt has a positive rate in terms of LOC and 42.2\% of technical debt has it in terms of Fan-In.

\smallsection{Future work} This paper shows only early idea to quantity interest of SATD. Therefore, there remain
many challenges in this topic. 

\begin{itemize}
\item To calculate interest, we use the relative size of metric values between two versions of SATD-introduction and reduction. However, the period is not considered to calculate the interest. Therefore, we would like to take the period into account when calculating the interest.
\item  There are several type of technical debt such as defect technical debt and design technical debt.
The previous study~\cite{Maldonado2015MTD} shows that the percentage of technical debt varies depending on the type of technical debt and the studied systems. For example, the projects that have limited time to develop features are likely to leave comments of features that need to be implemented in the future. 
To better understand the interest, we would like to analyze the interest per type of technical debt.
\item  The interest varies among technical debt. If we can understand the reason why some of technical debt has large interest, we can make use of such insights for future development. Therefore, we would like to manually investigate why some of technical debt has large interest.
\item Generally speaking, software systems are always evolving over time for implementing new functionality and fixing defects.
Therefore, even if the size of technical debt increases, it is not clear about how the nature of software evaluation affects the interest of technical debt.
We would like to compare the impact of software evolution on methods in two groups of SATD v.s. non-SATD.
%\item To operationalize our findings, we also built a tool that is able to identify and assign an interest rate to all SATD instances in a project. Our tool is publicly available and can be used by practitioners to prioritize the most impacting (i.e., highest interest) SATD.
\end{itemize}

\para{Summary of this work.}

\section*{Acknowledgment}
This research was partially supported by JSPS KAKENHI Grant Numbers 15H05306.

\bibliographystyle{abbrv}
%\bibliographystyle{IEEEtranN}
\bibliography{reference}


\end{document}

