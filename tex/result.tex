\section{Results} \label{sec:results}
\subsection{RQ1: Can we quantify interest of technical debt at the method-level?}
\smallsection{Motivation}
To alleviate the impact of technical debt, there are several previous studies on understanding SATD (e.g., the detection of technical debt~\cite{Potdar2014ICSME,Zazworka2013EASE} and the impact of SATD on software quality~\cite{Wehaibi2016SANER}).
However, there are few empirical studies that quantify interest of SATD.
Therefore, we would like to know how we can understand interest using our method explained in Section \ref{subsec:interest}

\smallsection{Approach}
%\para{We calculate the interest.}
To calculate interest of SATD, we follow the approach we explained in Section \ref{sec:setup}.

\smallsection{Results}
We find that there are high correlations between LOC and the other product metrics except Fan-In. 
From the highly correlated metrics, we choose LOC as metrics to calculate interest, similar to previous work that considers effort in the domain of defect prediction~\cite{Kamei2010ICSM,Kamei2013TSE}. We assume that developers spend more effort to check larger methods before modifying the methods. Eventually, we show our results using two product metrics (i.e., LOC and Fan-In).

Table \ref{tab:statistic} shows the number of SATD and the percentage of the technical debt that has positive interest in all technical debt. 

Figure \ref{fig:dist} shows that the distribution of interest. X
\para{Put the distribution of interest?}

\begin{table}[tb]
  \caption{Statistic interest of SATD}
  \label{tab:statistic}
  \centering

  \begin{tabular}{cl|r|rrr}
  \hline
      &  Project & Positive Rate & All & Positive & Negative \\
  \hline
        & Ant    & 38.0\% &  71 &  27  &  20 \\
   LOC  & JMeter & 44.2\% & 181 &  80  &  25 \\
        & JRuby  & 32.6\% & 236 &  77  &  59 \\
  \hline
        & Ant    & 30.9\% &  68 &  21  &  13 \\
Fan-In  & JMeter & 42.2\% & 161 &  68  &  13 \\
        & JRuby  & 30.3\% & 231 &  70  &  37 \\
  \hline
  \end{tabular}
\end{table}

%-----------------------------------------------------------------------
\begin{figure*}[!t]
  \begin{center}
  \scalebox{0.95}{
  \begin{tabular}{ccc}
    \subfigure[Ant (LOC)]{ 
      \includegraphics[width=.33\textwidth]{figures/rq1-ant}
    }
    \subfigure[JMeter (LOC)]{ 
      \includegraphics[width=.33\textwidth]{figures/rq1-jmeter}
    }
    \subfigure[JRuby (LOC)]{
      \includegraphics[width=.33\textwidth]{figures/rq1-jruby}
    }
  \end{tabular}
  }
  \scalebox{0.95}{
  \begin{tabular}{ccc}
    \subfigure[Ant (Fan-In)]{ 
      \includegraphics[width=.33\textwidth]{figures/rq1-ant-fanin}
    }
    \subfigure[JMeter (Fan-In)]{ 
      \includegraphics[width=.33\textwidth]{figures/rq1-jmeter-fanin}
    }
    \subfigure[JRuby (Fan-In)]{
      \includegraphics[width=.33\textwidth]{figures/rq1-jruby-fanin}
    }
  \end{tabular}
  }
  \caption{(RQ1) The results of distribution of interest.}
  \label{fig:dist}
  \end{center}
\end{figure*}
%-----------------------------------------------------------------------


\conclusionbox{Result of RQ1: 22\%-36\% of technical debt has positive interest.}

\subsection{RQ2: Does the interest differ based on the type of technical debt?}
\smallsection{Motivation}
There are several type of technical debt such as defect technical debt and design technical debt. 
To better understand the interest, we would like to analayze ...

\smallsection{Approach}
We classify technical debt into categories and calculate interest in each category.

\smallsection{Results}
Table X shows the number and the percentage of technical debt. Among three projects, jruby only includes more than one category that has more than 10\% of technical debt methods. Therefore, we decided to use jruby in RQ2.

\para{we report same things in RQ1.}  

%-----------------------------------------------------------------------
\begin{figure}
  \centering
  \includegraphics[width=.45\textwidth]{figures/rq2-defect}
  \caption{Adding a Repository in Commit Guru \label{fig:guru1}}
\end{figure}
\begin{figure}
  \centering
  \includegraphics[width=.45\textwidth]{figures/rq2-design}
  \caption{Adding a Repository in Commit Guru \label{fig:guru1}}
\end{figure}
\begin{figure}
  \centering
  \includegraphics[width=.45\textwidth]{figures/rq2-requirement}
  \caption{Adding a Repository in Commit Guru \label{fig:guru1}}
\end{figure}
%-----------------------------------------------------------------------


\conclusionbox{Result of RQ2}

\subsection{RQ3: manual analysis}
\smallsection{Motivation}

\smallsection{Approach}

\smallsection{Results}

\conclusionbox{Result of RQ3}

\section{Discussion}
\para{Put additional analysis when considering time period.}

\para{Put additional analysis when considering other metrics (fan-in).}

\para{Put additional analysis when comparing the non-technical deb methods.}

\para{Put the analysis for showing the method that includes more than one technical debt in one version.}