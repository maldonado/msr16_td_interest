\section{Conclusion} \label{conclusion}

% SDP has been well researched
The field of software defect prediction has been well-reserached since it was first proposed. As our paper showed, there have been many papers that explored different types of data and their implications, proposed various metrics, examined the applicability of different modeling techniques and evaluation criteria.

% Some of the work here has made a series of accomplishments - we should summarize them here
The field of software defect prediction has made a series of accomplishments. As our paper highlighted, there have been many papers that alleviate the challenges about data (e.g., the recent trend of data sharing and openness have in many ways helped alleviate the challenge that existed in the early 2000s), metrics (e.g., studies have used defect prediction to examine the impact of certain phenomena, e.g., ownership, on code quality), model building (e.g., thanks to cross-project defect prediction models, the recent defect prediction models are now available for the projects in the initial development phases) and model evaluation (e.g., we see a very healthy and progressive trend where software defect prediction studies are becoming more transparent.).

% At the same time, particular initiavites and works have had a profound impact on the area of SDP, which we list as game changers - should summarize them here.
At the same time, particular initiatives and works have had a profound impact on the area of SDP, which we listed as game changers. OSS projects provide rich, extensive, and readily available software repositories. The PROMISE repository provide SE researchers some motivation to share the dataset each researcher made. The SZZ algorithm automatically extracts whether or not a change introduces a defect from VCSs and dramatically accelerates the research speed of JIT defect prediction. Weka and R provide a wide variety of data pre-processing, statistical and support for graphical techniques.

% That said, there remain many future challenges for the field of which we tried to highlight some.
That said, there remain many future challenges for the field of which we tried to highlight. 
For example, we need to tackle that the generality of our findings and techniques to non open source software projects (e.g., commercial projects) is not studied in depth. We also need to consider new market (e.g., mobile applications and energy consumption) in the domain of software quality assurance.

% This paper only provides the authors perspective based on their preferences and expereinces. (we should say what we hope the paper provides to the readers here)
This paper only provides the authors perspective based on their preferences and experiences.
We hope that the paper provides the readers an understanding of whole pictures of defect prediction studies and some key challenges for future and delivers the discussion of the future of software quality assurance. 
%\todo{Next version.} 